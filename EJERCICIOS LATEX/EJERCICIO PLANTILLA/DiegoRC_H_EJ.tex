\documentclass[aspectratio=169]{beamer}

% Tema y configuración
\usetheme{Madrid}
\usecolortheme{default}

% Paquetes necesarios
\usepackage[utf8]{inputenc}
\usepackage[spanish]{babel}
\usepackage{amsmath}
\usepackage{amsfonts}
\usepackage{amssymb}
\usepackage{graphicx}
\usepackage{listings}
% Habilitar UTF-8 dentro de lstlisting (acentos, ñ, etc.)
\usepackage{listingsutf8}
\usepackage{xcolor}
\usepackage{verbatim}
% Para dibujos con TikZ
\usepackage{tikz}
\usetikzlibrary{positioning,trees}

% Configuración de listings para código
\lstset{
    basicstyle=\ttfamily\footnotesize,
    keywordstyle=\color{blue}\bfseries,
    commentstyle=\color{green},
    stringstyle=\color{red},
    showstringspaces=false,
    breaklines=true,
    frame=single,
    numbers=left,
    numberstyle=\tiny\color{gray},
    inputencoding=utf8,
    % Mapeo de caracteres españoles para que no falle la compilación
    literate={á}{{\'a}}1 {é}{{\'e}}1 {í}{{\'\i}}1 {ó}{{\'o}}1 {ú}{{\'u}}1
             {Á}{{\'A}}1 {É}{{\'E}}1 {Í}{{\'I}}1 {Ó}{{\'O}}1 {Ú}{{\'U}}1
             {ñ}{{\~n}}1 {Ñ}{{\~N}}1 {ü}{{\"u}}1 {Ü}{{\"U}}1
}

% Información del documento
\title{Hoja 0: Ejercicio 0}
\subtitle{TEMA}
\author{Diego Rodríguez Cubero}
\institute{UCM}
\date{\today}

% Logo (opcional)
% \logo{\includegraphics[height=1cm]{images/logo.png}}

\begin{document}

% Diapositiva de título
\begin{frame}
    \titlepage
\end{frame}

% Tabla de contenidos
\begin{frame}{Contenidos}
    \tableofcontents
\end{frame}

% Sección 1
\section{Enunciado del problema}
\begin{frame}{Enunciado del problema}
    \begin{itemize}
        \item Primer punto importante
        \item Segundo punto con \textbf{énfasis}
        \item Tercer punto con \textit{cursiva}
        \item Cuarto punto con \textcolor{red}{color}
    \end{itemize}
\end{frame}

\begin{frame}{Marco Teórico}
    \begin{block}{Definición}
        Aquí puedes colocar una definición importante.
    \end{block}
    
    \begin{alertblock}{Atención}
        Información que requiere atención especial.
    \end{alertblock}
    
    \begin{exampleblock}{Ejemplo}
        Un ejemplo práctico de la teoría.
    \end{exampleblock}
\end{frame}

% Sección 2
\section{Desarrollo}

\begin{frame}{Matemáticas}
    \begin{itemize}
        \item Ecuación en línea: $E = mc^2$
        \item Ecuación centrada:
        \begin{equation}
            \sum_{i=1}^{n} x_i = \frac{n(n+1)}{2}
        \end{equation}
        \item Sistema de ecuaciones:
        \begin{align}
            x + y &= 5 \\
            2x - y &= 1
        \end{align}
    \end{itemize}
\end{frame}

\begin{comment}
\begin{frame}[fragile]{Código}
    \begin{lstlisting}[language=Python, caption=Ejemplo de código Python]
def fibonacci(n):
    """Calcula el n-ésimo número de Fibonacci"""
    if n <= 1:
        return n
    else:
        return fibonacci(n-1) + fibonacci(n-2)

# Ejemplo de uso
print(fibonacci(10))
    \end{lstlisting}
\end{frame}
\end{comment}


\begin{frame}{Imágenes}
    \begin{figure}
        \centering
        % Descomenta la siguiente línea y coloca tu imagen en la carpeta images/
        % \includegraphics[width=0.6\textwidth]{images/ejemplo.png}
        \caption{Ejemplo de imagen (sustituir por imagen real)}
    \end{figure}
\end{frame}

% Sección 3
\section{Resultados}

\begin{frame}{Tabla de Resultados}
    \begin{table}
        \centering
        \begin{tabular}{|c|c|c|}
            \hline
            \textbf{Algoritmo} & \textbf{Tiempo (ms)} & \textbf{Precisión} \\
            \hline
            Algoritmo A & 150 & 95\% \\
            \hline
            Algoritmo B & 200 & 98\% \\
            \hline
            Algoritmo C & 120 & 92\% \\
            \hline
        \end{tabular}
        \caption{Comparación de algoritmos}
    \end{table}
\end{frame}

\begin{frame}{Gráficos}
    \begin{center}
        % Aquí podrías incluir gráficos generados con TikZ o imágenes
        \textit{Espacio reservado para gráficos}
    \end{center}
\end{frame}

% Sección 4
\section{Conclusiones}

\begin{frame}{Conclusiones}
    \begin{enumerate}
        \item Primera conclusión importante
        \item Segunda conclusión con implicaciones
        \item Tercera conclusión y trabajo futuro
    \end{enumerate}
    
    \vspace{1cm}
    
    \begin{center}
        \large \textbf{¡Gracias por su atención!}
    \end{center}
\end{frame}

% Referencias (opcional)
\begin{frame}{Referencias}
    \begin{thebibliography}{99}
        \bibitem{ref1} Autor, A. (2023). \textit{Título del artículo}. Revista Científica, 1(1), 1-10.
        \bibitem{ref2} Autor, B. (2022). \textit{Título del libro}. Editorial Académica.
    \end{thebibliography}
\end{frame}

\end{document}