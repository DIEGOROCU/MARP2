% Archivo de configuración de estilo personalizado
% Para usar este archivo, agrega % Archivo de configuración de estilo personalizado
% Para usar este archivo, agrega % Archivo de configuración de estilo personalizado
% Para usar este archivo, agrega % Archivo de configuración de estilo personalizado
% Para usar este archivo, agrega \input{config/estilo.tex} después de \documentclass en presentacion.tex

% Configuración de colores personalizados
\definecolor{azulUni}{RGB}{0, 51, 102}
\definecolor{grisClaro}{RGB}{240, 240, 240}
\definecolor{verdeExito}{RGB}{76, 175, 80}
\definecolor{rojoAlerta}{RGB}{244, 67, 54}

% Configuración del tema personalizado (opcional)
% \setbeamercolor{structure}{fg=azulUni}
% \setbeamercolor{palette primary}{bg=azulUni,fg=white}
% \setbeamercolor{palette secondary}{bg=azulUni!80,fg=white}
% \setbeamercolor{palette tertiary}{bg=azulUni!60,fg=white}

% Configuración de fuentes
% \setbeamerfont{title}{size=\Large,series=\bfseries}
% \setbeamerfont{frametitle}{size=\large,series=\bfseries}

% Configuración de la numeración de diapositivas
\setbeamertemplate{footline}[frame number]

% Eliminar iconos de navegación
\setbeamertemplate{navigation symbols}{}

% Configuración de itemize
\setbeamertemplate{itemize items}[circle]
\setbeamercolor{itemize item}{fg=azulUni}

% Configuración de bloques
\setbeamercolor{block title}{bg=azulUni,fg=white}
\setbeamercolor{block body}{bg=grisClaro,fg=black}

\setbeamercolor{block title alerted}{bg=rojoAlerta,fg=white}
\setbeamercolor{block body alerted}{bg=rojoAlerta!10,fg=black}

\setbeamercolor{block title example}{bg=verdeExito,fg=white}
\setbeamercolor{block body example}{bg=verdeExito!10,fg=black}

% Configuración de código (complementa listings)
\lstdefinestyle{miestilo}{
    backgroundcolor=\color{grisClaro},
    commentstyle=\color{green!60!black},
    keywordstyle=\color{blue}\bfseries,
    numberstyle=\tiny\color{gray},
    stringstyle=\color{orange},
    basicstyle=\ttfamily\footnotesize,
    breakatwhitespace=false,
    breaklines=true,
    captionpos=b,
    keepspaces=true,
    numbers=left,
    numbersep=5pt,
    showspaces=false,
    showstringspaces=false,
    showtabs=false,
    tabsize=2
}

% Lenguaje pseudocódigo (usa miestilo como base)
\lstdefinelanguage{pseudocode}{
    morekeywords=[1]{si, entonces, sino, finsi, mientras, hacer, finmientras,
                     para, desde, hasta, finpara, retornar, funcion, fin,
                     inicio, verdadero, falso, y, o, no,
                     if, then, else, endif, while, do, endwhile,
                     for, to, endfor, return, function, begin, end,
                     true, false, and, or, not, null},
    morekeywords=[2]{entero, real, booleano, cadena, lista, array,
                     int, float, bool, string, void},
    sensitive=false,
    morecomment=[l]{//},
    morecomment=[s]{/*}{*/},
    morestring=[b]",
    morestring=[b]',
}
\lstdefinestyle{pseudostyle}{
    style=miestilo,
    language=pseudocode,
    keywordstyle=[1]{\color{blue}\bfseries},
    keywordstyle=[2]{\color{teal}\bfseries},
    commentstyle=\color{gray}\itshape,
    literate={á}{\'a}1 {é}{\'e}1 {ó}{\'o}1 {ú}{\'u}1 {í}{\'\i}1
             {ñ}{\~n}1 {←}{{$\leftarrow$}}1 {≤}{{$\leq$}}1
             {≥}{{$\geq$}}1 {≠}{{$\neq$}}1
}

% Aplicar el estilo por defecto
\lstset{style=miestilo}

% Comando personalizado para resaltar texto importante
\newcommand{\importante}[1]{\textcolor{rojoAlerta}{\textbf{#1}}}

% Comando para notas al pie más elegantes
\newcommand{\fuente}[1]{\footnotetext{\tiny #1}}

% Configuración para ecuaciones numeradas
\setbeamertemplate{theorems}[numbered] después de \documentclass en presentacion.tex

% Configuración de colores personalizados
\definecolor{azulUni}{RGB}{0, 51, 102}
\definecolor{grisClaro}{RGB}{240, 240, 240}
\definecolor{verdeExito}{RGB}{76, 175, 80}
\definecolor{rojoAlerta}{RGB}{244, 67, 54}

% Configuración del tema personalizado (opcional)
% \setbeamercolor{structure}{fg=azulUni}
% \setbeamercolor{palette primary}{bg=azulUni,fg=white}
% \setbeamercolor{palette secondary}{bg=azulUni!80,fg=white}
% \setbeamercolor{palette tertiary}{bg=azulUni!60,fg=white}

% Configuración de fuentes
% \setbeamerfont{title}{size=\Large,series=\bfseries}
% \setbeamerfont{frametitle}{size=\large,series=\bfseries}

% Configuración de la numeración de diapositivas
\setbeamertemplate{footline}[frame number]

% Eliminar iconos de navegación
\setbeamertemplate{navigation symbols}{}

% Configuración de itemize
\setbeamertemplate{itemize items}[circle]
\setbeamercolor{itemize item}{fg=azulUni}

% Configuración de bloques
\setbeamercolor{block title}{bg=azulUni,fg=white}
\setbeamercolor{block body}{bg=grisClaro,fg=black}

\setbeamercolor{block title alerted}{bg=rojoAlerta,fg=white}
\setbeamercolor{block body alerted}{bg=rojoAlerta!10,fg=black}

\setbeamercolor{block title example}{bg=verdeExito,fg=white}
\setbeamercolor{block body example}{bg=verdeExito!10,fg=black}

% Configuración de código (complementa listings)
\lstdefinestyle{miestilo}{
    backgroundcolor=\color{grisClaro},
    commentstyle=\color{green!60!black},
    keywordstyle=\color{blue}\bfseries,
    numberstyle=\tiny\color{gray},
    stringstyle=\color{orange},
    basicstyle=\ttfamily\footnotesize,
    breakatwhitespace=false,
    breaklines=true,
    captionpos=b,
    keepspaces=true,
    numbers=left,
    numbersep=5pt,
    showspaces=false,
    showstringspaces=false,
    showtabs=false,
    tabsize=2
}

% Lenguaje pseudocódigo (usa miestilo como base)
\lstdefinelanguage{pseudocode}{
    morekeywords=[1]{si, entonces, sino, finsi, mientras, hacer, finmientras,
                     para, desde, hasta, finpara, retornar, funcion, fin,
                     inicio, verdadero, falso, y, o, no,
                     if, then, else, endif, while, do, endwhile,
                     for, to, endfor, return, function, begin, end,
                     true, false, and, or, not, null},
    morekeywords=[2]{entero, real, booleano, cadena, lista, array,
                     int, float, bool, string, void},
    sensitive=false,
    morecomment=[l]{//},
    morecomment=[s]{/*}{*/},
    morestring=[b]",
    morestring=[b]',
}
\lstdefinestyle{pseudostyle}{
    style=miestilo,
    language=pseudocode,
    keywordstyle=[1]{\color{blue}\bfseries},
    keywordstyle=[2]{\color{teal}\bfseries},
    commentstyle=\color{gray}\itshape,
    literate={á}{\'a}1 {é}{\'e}1 {ó}{\'o}1 {ú}{\'u}1 {í}{\'\i}1
             {ñ}{\~n}1 {←}{{$\leftarrow$}}1 {≤}{{$\leq$}}1
             {≥}{{$\geq$}}1 {≠}{{$\neq$}}1
}

% Aplicar el estilo por defecto
\lstset{style=miestilo}

% Comando personalizado para resaltar texto importante
\newcommand{\importante}[1]{\textcolor{rojoAlerta}{\textbf{#1}}}

% Comando para notas al pie más elegantes
\newcommand{\fuente}[1]{\footnotetext{\tiny #1}}

% Configuración para ecuaciones numeradas
\setbeamertemplate{theorems}[numbered] después de \documentclass en presentacion.tex

% Configuración de colores personalizados
\definecolor{azulUni}{RGB}{0, 51, 102}
\definecolor{grisClaro}{RGB}{240, 240, 240}
\definecolor{verdeExito}{RGB}{76, 175, 80}
\definecolor{rojoAlerta}{RGB}{244, 67, 54}

% Configuración del tema personalizado (opcional)
% \setbeamercolor{structure}{fg=azulUni}
% \setbeamercolor{palette primary}{bg=azulUni,fg=white}
% \setbeamercolor{palette secondary}{bg=azulUni!80,fg=white}
% \setbeamercolor{palette tertiary}{bg=azulUni!60,fg=white}

% Configuración de fuentes
% \setbeamerfont{title}{size=\Large,series=\bfseries}
% \setbeamerfont{frametitle}{size=\large,series=\bfseries}

% Configuración de la numeración de diapositivas
\setbeamertemplate{footline}[frame number]

% Eliminar iconos de navegación
\setbeamertemplate{navigation symbols}{}

% Configuración de itemize
\setbeamertemplate{itemize items}[circle]
\setbeamercolor{itemize item}{fg=azulUni}

% Configuración de bloques
\setbeamercolor{block title}{bg=azulUni,fg=white}
\setbeamercolor{block body}{bg=grisClaro,fg=black}

\setbeamercolor{block title alerted}{bg=rojoAlerta,fg=white}
\setbeamercolor{block body alerted}{bg=rojoAlerta!10,fg=black}

\setbeamercolor{block title example}{bg=verdeExito,fg=white}
\setbeamercolor{block body example}{bg=verdeExito!10,fg=black}

% Configuración de código (complementa listings)
\lstdefinestyle{miestilo}{
    backgroundcolor=\color{grisClaro},
    commentstyle=\color{green!60!black},
    keywordstyle=\color{blue}\bfseries,
    numberstyle=\tiny\color{gray},
    stringstyle=\color{orange},
    basicstyle=\ttfamily\footnotesize,
    breakatwhitespace=false,
    breaklines=true,
    captionpos=b,
    keepspaces=true,
    numbers=left,
    numbersep=5pt,
    showspaces=false,
    showstringspaces=false,
    showtabs=false,
    tabsize=2
}

% Lenguaje pseudocódigo (usa miestilo como base)
\lstdefinelanguage{pseudocode}{
    morekeywords=[1]{si, entonces, sino, finsi, mientras, hacer, finmientras,
                     para, desde, hasta, finpara, retornar, funcion, fin,
                     inicio, verdadero, falso, y, o, no,
                     if, then, else, endif, while, do, endwhile,
                     for, to, endfor, return, function, begin, end,
                     true, false, and, or, not, null},
    morekeywords=[2]{entero, real, booleano, cadena, lista, array,
                     int, float, bool, string, void},
    sensitive=false,
    morecomment=[l]{//},
    morecomment=[s]{/*}{*/},
    morestring=[b]",
    morestring=[b]',
}
\lstdefinestyle{pseudostyle}{
    style=miestilo,
    language=pseudocode,
    keywordstyle=[1]{\color{blue}\bfseries},
    keywordstyle=[2]{\color{teal}\bfseries},
    commentstyle=\color{gray}\itshape,
    literate={á}{\'a}1 {é}{\'e}1 {ó}{\'o}1 {ú}{\'u}1 {í}{\'\i}1
             {ñ}{\~n}1 {←}{{$\leftarrow$}}1 {≤}{{$\leq$}}1
             {≥}{{$\geq$}}1 {≠}{{$\neq$}}1
}

% Aplicar el estilo por defecto
\lstset{style=miestilo}

% Comando personalizado para resaltar texto importante
\newcommand{\importante}[1]{\textcolor{rojoAlerta}{\textbf{#1}}}

% Comando para notas al pie más elegantes
\newcommand{\fuente}[1]{\footnotetext{\tiny #1}}

% Configuración para ecuaciones numeradas
\setbeamertemplate{theorems}[numbered] después de \documentclass en presentacion.tex

% Configuración de colores personalizados
\definecolor{azulUni}{RGB}{0, 51, 102}
\definecolor{grisClaro}{RGB}{240, 240, 240}
\definecolor{verdeExito}{RGB}{76, 175, 80}
\definecolor{rojoAlerta}{RGB}{244, 67, 54}

% Configuración del tema personalizado (opcional)
% \setbeamercolor{structure}{fg=azulUni}
% \setbeamercolor{palette primary}{bg=azulUni,fg=white}
% \setbeamercolor{palette secondary}{bg=azulUni!80,fg=white}
% \setbeamercolor{palette tertiary}{bg=azulUni!60,fg=white}

% Configuración de fuentes
% \setbeamerfont{title}{size=\Large,series=\bfseries}
% \setbeamerfont{frametitle}{size=\large,series=\bfseries}

% Configuración de la numeración de diapositivas
\setbeamertemplate{footline}[frame number]

% Eliminar iconos de navegación
\setbeamertemplate{navigation symbols}{}

% Configuración de itemize
\setbeamertemplate{itemize items}[circle]
\setbeamercolor{itemize item}{fg=azulUni}

% Configuración de bloques
\setbeamercolor{block title}{bg=azulUni,fg=white}
\setbeamercolor{block body}{bg=grisClaro,fg=black}

\setbeamercolor{block title alerted}{bg=rojoAlerta,fg=white}
\setbeamercolor{block body alerted}{bg=rojoAlerta!10,fg=black}

\setbeamercolor{block title example}{bg=verdeExito,fg=white}
\setbeamercolor{block body example}{bg=verdeExito!10,fg=black}

% Configuración de código (complementa listings)
\lstdefinestyle{miestilo}{
    backgroundcolor=\color{grisClaro},
    commentstyle=\color{green!60!black},
    keywordstyle=\color{blue}\bfseries,
    numberstyle=\tiny\color{gray},
    stringstyle=\color{orange},
    basicstyle=\ttfamily\footnotesize,
    breakatwhitespace=false,
    breaklines=true,
    captionpos=b,
    keepspaces=true,
    numbers=left,
    numbersep=5pt,
    showspaces=false,
    showstringspaces=false,
    showtabs=false,
    tabsize=2
}

% Lenguaje pseudocódigo (usa miestilo como base)
\lstdefinelanguage{pseudocode}{
    morekeywords=[1]{si, entonces, sino, finsi, mientras, hacer, finmientras,
                     para, desde, hasta, finpara, retornar, funcion, fin,
                     inicio, verdadero, falso, y, o, no,
                     if, then, else, endif, while, do, endwhile,
                     for, to, endfor, return, function, begin, end,
                     true, false, and, or, not, null},
    morekeywords=[2]{entero, real, booleano, cadena, lista, array,
                     int, float, bool, string, void},
    sensitive=false,
    morecomment=[l]{//},
    morecomment=[s]{/*}{*/},
    morestring=[b]",
    morestring=[b]',
}
\lstdefinestyle{pseudostyle}{
    style=miestilo,
    language=pseudocode,
    keywordstyle=[1]{\color{blue}\bfseries},
    keywordstyle=[2]{\color{teal}\bfseries},
    commentstyle=\color{gray}\itshape,
    literate={á}{\'a}1 {é}{\'e}1 {ó}{\'o}1 {ú}{\'u}1 {í}{\'\i}1
             {ñ}{\~n}1 {←}{{$\leftarrow$}}1 {≤}{{$\leq$}}1
             {≥}{{$\geq$}}1 {≠}{{$\neq$}}1
}

% Aplicar el estilo por defecto
\lstset{style=miestilo}

% Comando personalizado para resaltar texto importante
\newcommand{\importante}[1]{\textcolor{rojoAlerta}{\textbf{#1}}}

% Comando para notas al pie más elegantes
\newcommand{\fuente}[1]{\footnotetext{\tiny #1}}

% Configuración para ecuaciones numeradas
\setbeamertemplate{theorems}[numbered]